
% [Consiste na maneira de trabalhar o objeto de pesquisa, as ações pelas quais serão alcançados os resultados esperados da pesquisa.
% Descrever:
% \begin{itemize}
% 	\item processo de estudo das fontes bibliográficas
% 	\item Instrumentos e fontes para coleta de dados
% 	\item Processos e métodos para a realização do trabalho (uso de ferramentas, linguagens de especificação, descrição das fases do trabalho de pesquisa, etc)
% \end{itemize}
% ]

Para atingir nossos objetivos, iremos coletar os dados gerados pelas ferramentas
implantadas ao final de cada sprint da equipe de Manutenção e Evolução de Software.
Neste projeto, será utilizado Pylint e Pygenie, ferramentas que irão realizar a
análise estática do código.

Pylint checa por erros no código, tentando padronizar o código e ao mesmo tempo
procurar por \textit{bad smells}, ou seja, por trechos do código que ferem as boas
práticas de programação. O pylint segue as diretrizes do \textit{PEP 008}, que é
um guia de boas práticas para projetos escritos em \textit{Python}. Lembrando sempre
que o pylint não é algo a ser seguido "a ferro e fogo", e serve apenas como sugestão
para refatoração de código. Além do \textit{PEP 008}, o pylint também segue as
práticas sugeridas por Matin Fowler em seu livro \textit{Refactoring}.

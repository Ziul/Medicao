
\subsection{Contexto} % (fold)
\label{sub:contexto}
O alvo de estudo é um software para clínicas de psicologia, o GestorPsi.
Este software busca ser ao mesmo tempo generalista, ou seja,
pode adaptar-se à maioria dos contextos das clínicas de psicologia, e também,
adequado para esses negócios em suas particularidades.

Desde o início o software é mantido por apenas um desenvolvedor,
e este não se preocupou em escrever casos de teste para o software, o que,
indiretamente, faz com que muitos bugs sejam descobertos em tempo de produção.
Entretanto, é preciso saber que somente "testar por testar" o software não garante
que o produto será estável e/ou confiável.
\subsection{Formulação do problema} % (fold)
\label{sub:formula_o_do_problema}

A disciplina de Manutenção e Evolução de Software possui a proposta de fazer com
que os alunos do curso de Engenharia de Software contribuam com projetos reais
que sejam software livre.
Assim, os alunos tem a possibilidade de aplicar os conhecimentos adquiridos nas
diversas disciplinas cursadas com contribuições em forma de código.

O GestorPsi é um dos softwares utilizados na disciplina. Este não possui nem testes
automatizados, nem funcionais, o que o torna uma excelente base de aplicação de
técnicas de programação, também, o sistema não segue alguns \textit{code style guide}
da comunidade Python, como PEP 008 e o PEP 020.

A forma utilizada na disciplina para contribuição é através de \textit{fork}, ou seja,
através da obtenção de uma cópia do código-fonte da aplicação no \textit{forge}
(plataforma que abriga repositórios) que abriga o projeto, e após uma série de
refatorações e implementações de código, \textit{pull requests} com as alterações
feitas são submetidas para avaliação à equipe que mantém o sistema. Isso faz com
que haja diferenças entre o repositório oficial do sistema e o que está sendo
refatorado pelos alunos da Universidade de Brasília.

\subsection{Objetivos} % (fold)
\label{sub:objetivos}

\subsubsection{Objetivos Geral}

Acompanhar a evolução de um projeto de software livre com contribuições de alunos
da Engenharia de Software.

\subsubsection{Objetivos Específicos}
[ Responder à questão: O QUE FAZER?
Apresentar os objetivos como Objetivo Geral e Objetivos específicos.
Objetivo Geral: o que se pretende atingir / alcançar com a pesquisa
Objetivos específicos: São etapas do trabalho para se alcançar / atingir o objetivo geral. Utilizar verbos no infinitivo para descrever tais objetivos específicos.
\begin{itemize}
	\item Exploratórios (conhecer, identificar, levantar, descobrir)
	\item Descritivos (caracterizar, descrever, traçar, determinar, definir)
	\item Explicativos (analisar, avaliar, verificar, explicar, validar)   ]
\end{itemize}


\subsection{Justificativas} % (fold)
\label{sub:justificativas}

[Basicamente, essa seção deve responder à seguinte questão: POR QUE FAZER?
As justificativas consistem em uma descrição e argumentações sobre as razões e motivações da escolha do tema de projeto em questão, de maneira a esclarecer as razões pelas quais o presente projeto é importante.
Essa descrição/argumentação deve indicar:
\begin{itemize}
	\item A importância do tema a ser investigado.
	\item As possíveis contribuições do projeto.
	\item Relação do tema com outras pesquisas.
\end{itemize}

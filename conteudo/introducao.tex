
\subsection{Contexto} % (fold)
\label{sub:contexto}
O alvo de estudo é um software para clínicas de psicologia, o GestorPsi.
Este software busca ser ao mesmo tempo generalista, ou seja,
pode adaptar-se à maioria dos contextos das clínicas de psicologia, e também,
adequado para esses negócios em suas particularidades.

Desde o início o software é mantido por apenas um desenvolvedor,
e este não se preocupou em escrever casos de teste para o software, o que,
indiretamente, faz com que muitos bugs sejam descobertos em tempo de produção.

\subsection{Formulação do problema} % (fold)
\label{sub:formula_o_do_problema}

[Responder à questão: QUAIS AS QUESTÕES A SEREM RESOLVIDAS?
Detalhar o problema citado no Contexto e apresentar as questões específicas a que a presente pesquisa pretende responder ou resolver. São as questões a serem solucionadas.]
]

\subsection{Objetivos} % (fold)
\label{sub:objetivos}

[ Responder à questão: O QUE FAZER?
Apresentar os objetivos como Objetivo Geral e Objetivos específicos.
Objetivo Geral: o que se pretende atingir / alcançar com a pesquisa
Objetivos específicos: São etapas do trabalho para se alcançar / atingir o objetivo geral. Utilizar verbos no infinitivo para descrever tais objetivos específicos.
\begin{itemize}
	\item Exploratórios (conhecer, identificar, levantar, descobrir)
	\item Descritivos (caracterizar, descrever, traçar, determinar, definir)
	\item Explicativos (analisar, avaliar, verificar, explicar, validar)   ]
\end{itemize}


\subsection{Justificativas} % (fold)
\label{sub:justificativas}

[Basicamente, essa seção deve responder à seguinte questão: POR QUE FAZER?
As justificativas consistem em uma descrição e argumentações sobre as razões e motivações da escolha do tema de projeto em questão, de maneira a esclarecer as razões pelas quais o presente projeto é importante.
Essa descrição/argumentação deve indicar:
\begin{itemize}
	\item A importância do tema a ser investigado.
	\item As possíveis contribuições do projeto.
	\item Relação do tema com outras pesquisas.
\end{itemize}
